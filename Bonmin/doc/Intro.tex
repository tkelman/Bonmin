%%% Local Variables:
%%% mode: latex
%%% TeX-master: "Bonmin_UsersManual"
%%% End:

\html{

\begin{PageSummary}
\PageName{Introduction}
\PageSection{Types of problems solved}{MathBack}
\PageSection{Algorithms}{Algos}
\PageSection{Required third party code}{ThirdP}
\PageSection{Supported platforms}{Support}
\end{PageSummary}


\begin{quickref}
\quickcitation{An algorithmic framework for convex MINLP. Bonami et.al.}{\BetalLink}
\quickcitation{Algorithms and Software for Convex Mixed Integer Nonlinear Programs. Bonami, Kilinc, Linderoth}{\HotMINLPLink}
\quickcitation{An outer-approximation algorithm for a class of MINLP. M. Duran and I.E. Grossmann. Mathematical Programming}{\DGLink}
\quickcitation{Branch and bound experiments in convex nonlinear integer programming. O.K. Gupta and V. Ravindran.}{\GuptaLink}
\quickcitation{Solving MINLP by outer approximation.
  R. Fletcher and S. Leyffer. Mathematical Programming.}{\FLLink}
\quickcitation{An LP/NLP based branched and bound algorithm for convex MINLP optimization problems. I. Quesada and I.E. Grossmann.
   Computers and Chemical Engineering.}{\QGLink}\end{quickref}
}

\PageTitle{\latexhtml{Introduction}{\Bonmin}}{sec:Intro}
\Bonmin\ (Basic Open-source Nonlinear Mixed INteger programming)
is an open-source code for solving general MINLP (Mixed
Integer NonLinear Programming) problems.
 It is distributed on
\COINOR
\latexhtml{{(\tt www.coin-or.org)}}{}
under the CPL (Common Public
License). The CPL is a license approved by the
\footlink{http://www.opensource.org}{OSI},
(Open Source Initiative),
 thus \Bonmin\ is OSI
Certified Open Source Software.

There are several algorithmic choices that can be selected with \Bonmin.
{\tt B-BB} is a NLP-based branch-and-bound algorithm,
{\tt B-OA} is an
outer-ap\-prox\-i\-ma\-tion decomposition algorithm, {\tt B-iFP} is an iterated
feasibility pump algorithm, {\tt B-QG} is an
implementation of  Quesada and Grossmann's branch-and-cut algorithm,
{\tt B-Hyb} is a hybrid outer-ap\-prox\-i\-ma\-tion based
branch-and-cut algorithm and {\tt B-Ecp} is a variant of {\tt B-QG} based
on adding additional ECP cuts.


Some of the algorithmic choices require the ability to solve MILP
(Mixed Integer Linear Programming) problems and NLP (NonLinear
Programming) problems. The default solvers for these are,
respectively, the COIN-OR codes \Cbc\ and \Ipopt. In turn,
{\tt Cbc} uses further COIN-OR modules: \Clp\ (for LP (Linear
Programming) problems), \Cgl\ (for generating MILP cutting
planes), as well as various other utilities. It is also possible to
step outside the open-source realm and use
\Cplex\ as the MILP solver and FilterSQP as the NLP solver. 

Additional documentation can be found on the {\tt Bonmin}
\latex{ homepage at 
$$
         \hbox{\tt http://www.coin-or.org/Bonmin}
$$
 and wiki at 
$$
         \hbox{\tt https://projects.coin-or.org/Bonmin}
$$
}

\html{
\href{http://www.coin-or.org/Bonmin}{homepage} 
and \href{https://projects.coin-or.org/Bonmin}{wiki}.}

\subsectionHs{Types of problems solved}{MathBack}
\Bonmin\ solves MINLPs of the form

%\left\{
\begin{align*}
%\begin{array}{l}
&\min f(x) \\
& {\rm s.t.} \\
&g^L \leq g(x) \leq g^U,\\
& x^L \leq x \leq x^U, \\
&x \in \mathbb{R}^n, \;  x_i \in \mathbb{Z} \; \forall i \in I,
%\end{array}
\end{align*}
%\right.
where the functions $f :~\{x\in \mathbb{R}^n : x^L \leq x \leq x^U
\}~ \rightarrow~\mathbb{R}$ and $g:~\{x\in \mathbb{R}^n : x^L \leq x
\leq x^U \}~\rightarrow~\mathbb{R}^m$ are assumed to be twice
continuously differentiable, and $I \subseteq \{1, \ldots,n \}$. We
emphasize that \Bonmin\ treats problems that are cast
in {\em minimization} form.

The different methods that \Bonmin\ implements are exact algorithms when the functions $f$ and $g$ are
convex but are only heuristics when this is not the case (i.e., \Bonmin\ is not a \emph{global} optimizer).\\

\subsectionHs{Algorithms}{Algos}
\Bonmin\ implements six different algorithms for solving
MINLPs:

\begin{itemize}
\item {\tt B-BB}: a simple branch-and-bound algorithm based on solving
a continuous nonlinear program at each node of the search tree and
branching on variables \mycite{Gupta80Nonlinear}{Gupta 1980};
we also allow the possibility of SOS (Type 1) branching
\item {\tt B-OA}: an outer-approximation based decomposition algorithm
\latexhtml{\cite{DG,FL}}
{[\href{\DGLink}{Duran Grossmann 1986},\href{\FLLink}{Fletcher Leyffer 1994}]}
\item {\tt B-QG}: an outer-approximation based branch-and-cut
algorithm
\citeH{QG}{\QGLink}{Quesada Grossmann 1994}
\item {\tt B-Hyb}: a hybrid outer-approximation/nonlinear programming
based branch-and-cut algorithm \citeH{Betal}
{\BetalLink}{Bonami et al. 2008}
\item {\tt B-Ecp}: another outer-approximation based branch-and-cut inspired
by the settings described in \citeH{abhishek.leyffer.linderoth:06}{\AbhishekLink}{Abhishek Leyffer Linderoth 2006}
\item {\tt B-iFP}: an iterated feasibility pump algorithm \citeH{bonami.etal:06}{\FPLink}{Bonami Cornu\eacute jols Lodi Margot 2009}.
\end{itemize}

In this manual,  we will not go into a further description of these algorithms.
Mathematical details of these algorithms 
and some details of their implementations can be found in \citeH{Betal}{\BetalLink}{Bonami et al. 2008} and \citeH{hot:2009}{\HotMINLPLink}{Bonami K\i ln\i c Linderoth 2009}.

Whether or not you are interested in the details of the algorithms, you certainly
want to know which one of these six algorithms you should choose to solve
your particular problem.
For convex MINLPs, experiments we have made on a reasonably large test set of problems point in favor of using {\tt B-Hyb}
(it solved the most of the problems in our test set in 3 hours of computing time).
Nevertheless, there are cases where {\tt B-OA} is much faster than {\tt B-Hyb} and others where {\tt B-BB} is interesting.
{\tt B-QG} and {\tt B-ECP} correspond mainly to a specific parameter setting of {\tt B-Hyb} but they can be faster in some case. {\tt B-iFP} is more tailored at finding quickly good solutions to very hard convex MINLP.
For nonconvex MINLPs, we strongly recommend using {\tt B-BB} (the outer-approximation algorithms
have not been tailored to treat nonconvex problems at this point). Although even {\tt B-BB} is only a
heuristic for such problems, we have added several
options to try and improve the quality of the solutions it provides (see \latexhtml{Section \ref{sec:non_convex}}{\href{\OptSetPage \#sec:non_convex}{non convex options}}).
Because it is appliable to more classes problem {\tt B-BB} is the default algorithm in \Bonmin.

\subsectionHs{Required third party code}{ThirdP}
In order to run {\Bonmin}, you have to download other external
libraries (and pay attention to their licenses!):
\begin{itemize}
\item \href{\LapackAddr}{Lapack} (Linear Algebra
PACKage)
\item \href{\BlasAddr}{Blas} (Basic Linear Algebra
Subroutines)
\item a sparse linear solver that is supported by Ipopt, e.g., MA27 from the
\href{\AslAddr}{HSL}
(Harwell Subroutine Library), MUMPS, or Pardiso.
\end{itemize}

Note that Lapack and the Blas are free for commercial use from the
\footlink{http://www.netlib.org}{Netlib Repository}, but they are
not OSI Certified Open Source Software. The linear solver MA27 is
freely available for noncommercial use.

The above software is sufficient to run \Bonmin\ as a
stand-alone C++ code, but it does not provide a modeling language.
For functionality from a modeling language, \Bonmin\ can be
invoked from \footlink{http://www.ampl.com}{\tt Ampl} (no extra installation is required provided
that you have a licensed copy of {\tt Ampl} installed), though you
need the {\tt ASL} (Ampl Solver Library) which is obtainable from the Netlib.

\Bonmin\ can use FilterSQP \citeH{FilTer}{\FilterLink}{FletcherLeyffer1998} as an alternative to \Ipopt\ for solving NLPs.

Also, in the outer approximation methods {\tt B-OA} and {\tt B-iFP}, some MILP problems are
solved. By default \Bonmin\ uses  \Cbc\ to solve them, but it can also be set up to use
the commercial solver \footlink{http://www.cplex.com}{\Cplex}.

\subsectionHs{Tested platforms}{Support}
\Bonmin\ has been installed on the following systems:
\begin{itemize}
\item Linux using g++ version 3.* and 4.* until 4.6 and Intel 9.* and 10.*
\item Windows using version Cygwin 1.5.18
\item Mac OS X using gcc 3.* and 4.* until 4.3 and Intel 9.* and 10.*
\item SunOS 5 using gcc 4.3
\end{itemize}

