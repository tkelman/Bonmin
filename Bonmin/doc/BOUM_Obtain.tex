\PageHead{Downloading \Bonmin}
\begin{PageSummary}
\PageSection{Obtaining \Bonmin}{sec:obtain}
\PageSection{Obtaining required third party code}{sec:obtain_3rd}
\end{PageSummary}

\begin{quickref}
\quickcitation{Bonmin Wiki Pages}{\linkCoin Bonmin}
\quickcitation{subversion web page}{http://subversion.tigris.org/}
\quickcitation{Using subversion on windows}{http://www.coin-or.org/faqs.html\#q4}
\quickcitation{Linear Algebra PACKage}{http://www.netlib.org/lapack/}
\quickcitation{Basic Linear Algebra Subroutines}{http://www.netlib.org/blas/}
\quickcitation{Harwell Subroutine Library}{http://www.cse.clrc.ac.uk/nag/hsl/contents.shtml}
\quickcitation{Ampl Solver Library}{http://www.ampl.com}
\end{quickref}

\PageTitle{Obtaining \Bonmin}{sec:obtain}



The \Bonmin\ package consists of the source code for the \Bonmin\
project but also source code from other \COINOR\ projects:
\begin{itemize}
\item \BuildTools
\item \Cbc
\item \Cgl
\item \Clp
\item \CoinUtils
\item \Ipopt
\item \Osi
\end{itemize}

When downloading the \Bonmin\ package you will download the source code for all these and
libraries of problems to test the codes.\\


You can obtain the \Bonmin\ package by using
\link{http://subversion.tigris.org/}{subversion}.

In Unix\footnote{UNIX is a registered trademark of The Open
Group.}-like environments, to download the code in a sub-directory, say {\tt coin-Bonmin} issue the following
command:
\break

\begin{colorverb}
 \noindent {\tt  svn co
https://projects.coin-or.org/svn/Bonmin/trunk~coin-Bonmin }
\end{colorverb}

This copies all the necessary COIN-OR files to compile \Bonmin\ to
{\tt coin-Bonmin}. To download \Bonmin\ using svn on Windows,
follow the instructions provided at
\link{http://www.coin-or.org/faqs.html\#q4}{COIN-OR}.

\subsectionH{Obtaining required third party code}{sec:obtain_3rd}
\Bonmin\ needs a few external packages which are not included in the \Bonmin\ package:
\begin{itemize}
\item Lapack (Linear Algebra PACKage)
\item Blas (Basic Linear Algebra Subroutines)
\item the sparse linear solver MA27 from the Harwell Subroutine Library and optionally (but strongly recommended) MC19 to enable automatic scaling in \Ipopt.
\item optionally ASL (the Ampl Solver Library), to be able to use \Bonmin\ from Ampl.
\end{itemize}

Since these third-party software modules are released under licenses
that are incompatible with the CPL, they cannot be included for
distribution with \Bonmin\ from COIN-OR, but you will find scripts
to help you download them in the subdirectory {\tt ThirdParty} of
the \Bonmin\ distribution\footnote{In most Linux distribution and
CYGWIN, Lapack and Blas are available as prebuilt binary packages in
the distribution (and are probably already installed on your
machine).}. For details on how to obtain these package, refer to the
instructions in
\link{http://www.coin-or.org/Ipopt/documentation/node13.html}{Section 2.2} of the Ipopt manual.\\
