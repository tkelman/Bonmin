\PageHead{Running \Bonmin}
\StartPageSummary
\PageSection{On an .nl file}{sec:run_nl}
\PageSection{From Ampl}{sec:run_ampl}
\PageSection{From a C++ Programm}{sec:run_cpp}
\EndPageSummary
\NavigationPanel



\PageTitle{Running \Bonmin}{sec:run}
\Bonmin\ can be run
\begin{itemize}
\item [(i)] from a command line on a {\tt .nl} file
(see \cite{Gay}),
\item [(ii)] from the modeling language \footlink{http://www.ampl.com}{\tt Ampl} (see
\cite{AMPL}) or from \footlink{http://www.gams.com/}{Gams} provided
that you have a valid {\tt Ampl} license, and
\item [(iii)] by invoking it from a C/C++ program.
\end{itemize}
Eventually, we expect that a fourth option will be remotely, via the
\footlink{http://neos.mcs.anl.gov/neos}{NEOS} Server for Optimization.
In the subsections that follow, we give some details about the
various ways to run \Bonmin.

\subsectionH{On a {\tt .nl} file}{sec:run_nl}
\Bonmin\ can read a {\tt .nl} file which could be generated by {\tt
Ampl} (for example {\tt mytoy.nl} in the {\tt
Bonmin-dist/Bonmin/test} subdirectory). The command line takes just
one argument which is the name of the {\tt .nl} file to be
processed.

For example, if you want to solve {\tt mytoy.nl}, from the {\tt
Bonmin-dist} directory, issue the command:

\begin{verbatim}
bonmin test/mytoy.nl
\end{verbatim}

\subsectionH{From {\tt Ampl}}{sec:run_ampl}
To use \Bonmin\ from {\tt Ampl} you just need to have the directory where the
{\tt bonmin} executable is in your {\tt \$PATH} and to issue the
command

\begin{verbatim}

option solver bonmin;

\end{verbatim}
in the {\tt Ampl} environment. Then the next {\tt solve} will
use \Bonmin\ to solve the model loaded in {\tt Ampl}.
After the optimization is finished, the values of the variables in the best-known
or optimal solution can be accessed in {\tt Ampl}. If the optimization is interrupted
with {\tt <CTRL-C>} the best known solution is accessible (this feature is not available in Cygwin).\\

A simple {\tt Ampl} example model follows:

\begin{verbatim}

   # An Ampl version of toy

   reset;

   var x binary;
   var z integer >= 0 <= 5;
   var y{1..2} >=0;
   minimize cost:
       - x - y[1] - y[2] ;

   subject to
       c1: ( y[1] - 1/2 )^2 + (y[2] - 1/2)^2 <= 1/4 ;
       c2: x - y[1] <= 0 ;
       c3: x + y[2] + z <= 2;

   option solver bonmin; # Choose BONMIN as the solver (assuming that
                         # bonmin is in your PATH

   solve;                # Solve the model
   display x;
   display y;

\end{verbatim}
(This example can be found in the subdirectory {\tt Bonmin/examples/amplExamples/} of
the \Bonmin\ package.)

Branching priorities, branching directions and pseudo-costs can be passed using {\tt Ampl} suffixes.
The suffix for branching priorities is {\tt "priority"} (variables with a higher priority
will be chosen first for branching),
for branching direction is {\tt "direction"} (if direction is $1$ the $\geq$ branch
is explored first, if direction is $-1$ the $\leq$ branch is explored first), for up
and down pseudo costs {\tt "upPseudoCost"} and {\tt "downPseudoCost"} respectively
(note that if only one of the up and down pseudo-costs is set in the {\tt Ampl} model it will
be used for both up and down).\\

For example, to give branching priorities of $10$ to variables {\tt y} and 1 to variable {\tt x}
and to set the branching directions to explore the upper branch first for all variables
in the simple example given, we add before the call to solve:
\begin{verbatim}

suffix priority IN, integer, >=0, <= 9999;
y[1].priority := 10;
y[2].priority := 10;
x.priority := 1;

suffix direction IN, integer, >=-1, <=1;
y[1].direction := 1;
y[2].direction := 1;
x.direction := 1;

\end{verbatim}

SOS Type-1 branching is also available in \Bonmin\ from {\tt Ampl}. We
follow the conventional way of doing this with suffixes.
Two type of suffixes should be declared:

\begin{verbatim}
suffix sosno IN, integer, >=1;  # Note that the solver assumes that these
                                #   values are positive for SOS Type 1
suffix ref IN;
\end{verbatim}

Next, suppose that we wish to have variables
\begin{verbatim}
var X {i in 1..M, j in 1..N} binary;
\end{verbatim}
and the ``convexity'' constraints:

\begin{verbatim}
subject to Convexity {i in 1..M}:
   sum {j in 1..N} X[i,j] = 1;
\end{verbatim}
(note that we must explicitly include the convexity constraints in the {\tt Ampl} model).

Then after reading in the data, we set the suffix values:
\begin{verbatim}

# The numbers `val[i,j]' are chosen typically as
#     the values `represented' by the discrete choices.
let {i in 1..M, j in 1..N} X[i,j].ref := val[i,j];

# These identify which SOS constraint each variable belongs to.
let {i in 1..M, j in 1..N} X[i,j].sosno := i;
\end{verbatim}

\subsubsection{From {\tt Gams} using {\tt Ampl}}
The modeling system {\tt Gams} offers the possibility to solve {\tt Gams} models using any
{\tt Ampl} capable solver. The {\tt Gams}/{\tt Ampl} link comes free with the {\tt Gams}
system, but users must have a licensed {\tt Ampl} on their machine.

To be able to use \Bonmin\ with {\tt Gams}, you must compile the COIN-OR libraries as static.
This can be done by passing the option {\tt --enable-static} to the configure file.

Detailed instruction for using {\tt Gams}/{\tt Ampl} can be found
on \footlink{http://www.gams.com/solvers/gamsampl.pdf}{{\tt Gams} website}.\\

Note that user set branching priorities and SOS Type-1 branching is not
available from this interface.

\subsectionH{From a C/C++ program}{sec:run_cpp}
\Bonmin\ can also be run from within a C/C++ program if the user codes
the functions to compute first- and second-order derivatives.
An example of such a program is available in the subdirectory {\tt CppExample} of
the {\tt examples} directory. For further explanations, please refer to the reference manual.
