\documentclass{article}
\usepackage[colorlinks=true]{hyperref}
\usepackage{html}
\usepackage{lscape}
\usepackage{amsmath,amssymb}
\usepackage{threeparttable}

\usepackage{bonmin}


\title{\Bonmin\ Users' Manual}
\author{Pierre Bonami and Jon Lee}


\begin{document}
\maketitle

%%% Local Variables:
%%% mode: latex
%%% TeX-master: "Bonmin_UsersManual"
%%% End:


\PageHead{Introduction}
\begin{PageSummary}
\PageSection{Types of problems solved}{MathBack}
\PageSection{Algorithms}{Algos}
\PageSection{Required third party code}{ThirdP}
\PageSection{Supported platforms}{Support}
\end{PageSummary}


\begin{quickref}
\quickcitation{An algorithmic framework for convex MINLP. Bonami et.al.}{\BetalLink}
\quickcitation{An outer-approximation algorithm for a class of MINLP. M. Duran and I.E. Grossmann. Mathematical Programming}{\DGLink}
\quickcitation{Branch and bound experiments in convex nonlinear integer programming. O.K. Gupta and V. Ravindran.}{\GuptaLink}
\quickcitation{Solving MINLP by outer approximation.
  R. Fletcher and S. Leyffer. Mathematical Programming.}{\FLLink}
\quickcitation{An LP/NLP based branched and bound algorithm for convex MINLP optimization problems. I. Quesada and I.E. Grossmann.
   Computers and Chemical Engineering.}{\QGLink}\end{quickref}

\PageTitle{\latexhtml{Introduction}{\Bonmin}}{sec:Intro}
\Bonmin\ (Basic Open-source Nonlinear Mixed INteger programming)
is an open-source code for solving general MINLP (Mixed
Integer NonLinear Programming) problems.
 It is distributed on
\COINOR
\latexhtml{\begin{latexonly} (\tt www.coin-or.org) \end{latexonly}}{}
under the CPL (Common Public
License). The CPL is a license approved by the
\footlink{http://www.opensource.org}{OSI},
(Open Source Initiative),
 thus \Bonmin\ is OSI
Certified Open Source Software.

There are several algorithmic choices that can be selected with \Bonmin.
{\tt B-BB} is a NLP-based branch-and-bound algorithm,
{\tt B-OA} is an
outer-ap\-prox\-i\-ma\-tion decomposition algorithm, {\tt B-QG} is an
implementation of  Quesada and Grossmann's branch-and-cut algorithm,
and {\tt B-Hyb} is a hybrid outer-ap\-prox\-i\-ma\-tion based
branch-and-cut algorithm.


Some of the algorithmic choices require the ability to solve MILP
(Mixed Integer Linear Programming) problems and NLP (NonLinear
Programming) problems. The default solvers for these are,
respectively, the COIN-OR codes \Cbc\ and \Ipopt. In turn,
{\tt Cbc} uses further COIN-OR modules: \Clp\ (for LP (Linear
Programming) problems), \Cgl\ (for generating MILP cutting
planes), as well as various other utilities. It is also possible to
step outside the open-source realm and use
\Cplex\ as the MILP solver. We expect to make an interface to other NLP solvers as well.

\subsectionHs{Types of problems solved}{MathBack}
\Bonmin\ solves MINLPs of the form

%\left\{
\begin{align*}
%\begin{array}{l}
&\min f(x) \\
& {\rm s.t.} \\
&g^L \leq g(x) \leq g^U,\\
& x^L \leq x \leq x^U, \\
&x \in \mathbb{R}^n, \;  x_i \in \mathbb{Z} \; \forall i \in I,
%\end{array}
\end{align*}
%\right.
where the functions $f :~\{x\in \mathbb{R}^n : x^L \leq x \leq x^U
\}~ \rightarrow~\mathbb{R}$ and $g:~\{x\in \mathbb{R}^n : x^L \leq x
\leq x^U \}~\rightarrow~\mathbb{R}^m$ are assumed to be twice
continuously differentiable, and $I \subseteq \{1, \ldots,n \}$. We
emphasize that \Bonmin\ treats problems that are cast
in {\em minimization} form.

The different methods that \Bonmin\ implements are exact algorithms when the functions $f$ and $g$ are
convex but are only heuristics when this is not the case (i.e., \Bonmin\ is not a \emph{global} optimizer).\\

\subsectionHs{Algorithms}{Algos}
\Bonmin\ implements four different algorithms for solving
MINLPs:

\begin{itemize}
\item {\tt B-BB}: a simple branch-and-bound algorithm based on solving
a continuous nonlinear program at each node of the search tree and
branching on variables \mycite{Gupta80Nonlinear}{Gupta80};
we also allow the possibility of SOS (Type 1) branching
\item {\tt B-OA}: an outer-approximation based decomposition algorithm
\latexhtml{\begin{latexonly} \cite{DG,FL} \end{latexonly}}
{[\link{\DGLink}{Duran1986},\link{\FLLink}{Leyffer1994}]}
\item {\tt B-QG}: an outer-approximation based branch-and-bound
algorithm
\citeH{QG}{\QGLink}{Quesada1994}
\item {\tt B-Hyb}: a hybrid outer-approximation/nonlinear programming
based branch-and-cut algorithm \citeH{Betal}
{\BetalLink}{Bonami2006}
\end{itemize}

In this manual,  we will not go into a further description of these algorithms.
Mathematical details of these algorithms
and some details of their implementations can be found in\citeH{Betal}{\BetalLink}{Bonami2006}.

Whether or not you are interested in the details of the algorithms, you certainly
want to know which one of these four algorithms you should choose to solve
your particular problem.
For convex MINLPs, experiments we have made on a reasonably large test set of problems point in favor of using {\tt B-Hyb}
(it solved the most of the problems in our test set in 3 hours of computing time).
Therefore, it is the default algorithm in \Bonmin.
Nevertheless, there are cases where {\tt B-OA} is much faster than {\tt B-Hyb} and others where {\tt B-BB} is interesting.
{\tt B-QG} corresponds mainly to a specific parameter setting of {\tt B-Hyb} where some features are disabled.
For nonconvex MINLPs, we strongly recommend using {\tt B-BB} (the outer-approximation algorithms
have not been tailored to treat nonconvex problems at this point). Although even {\tt B-BB} is only a
heuristic for such problems, we have added several
options to try and improve the quality of the solutions it provides (see \latexhtml{Section \ref{sec:non_convex}}{\link{\OptSetPage #sec:non_convex}{non convex options}}).

\subsectionHs{Required third party code}{ThirdP}
In order to run {\Bonmin}, you have to download other external
libraries (and pay attention to their licenses!):
\begin{itemize}
\item \link{\LapackAddr}{Lapack} (Linear Algebra
PACKage)
\item \link{\BlasAddr}{Blas} (Basic Linear Algebra
Subroutines)
\item the sparse linear solver MA27 from the
\link{\AslAddr}{HSL}
(Harwell Subroutine Library)
\end{itemize}

Note that Lapack and the Blas are free for commercial use from the
\footlink{http://www.netlib.org}{Netlib Repository}, but they are
not OSI Certified Open Source Software. The linear solver MA27 is
freely available for noncommercial use.

The above software is sufficient to run \Bonmin\ as a
stand-alone C++ code, but it does not provide a modeling language.
For functionality from a modeling language, \Bonmin\ can be
invoked from \footlink{http://www.ampl.com}{\tt Ampl} (no extra installation is required provided
that you have a licensed copy of {\tt Ampl} installed), though you
need the {\tt ASL} (Ampl Solver Library) which is obtainable from the Netlib.

Also, in the outer approximation decomposition method {\tt B-OA}, some MILP problems are
solved. By default \Bonmin\ uses  \Cbc\ to solve them, but it can also be set up to use
the commercial solver \footlink{http://www.ilog.com/products/cplex/product/mip.cfm}{\Cplex}.

\subsectionHs{Tested platforms}{Support}
\Bonmin\ has been installed on the following systems:
\begin{itemize}
\item Linux using g++ version 3.* and 4.*
\item Windows using version Cygwin 1.5.18
\item Mac OS X using gcc 3.* and 4.*
\end{itemize}

\latexhtml{}{
\begin{rawhtml}
</div>
</div>
<!--end content -->
<div id="siteInfo">
  <img src="" width="44" height="22"> <a href="#">About Us</a> | <a href="#">Contact Us</a> | &copy;2006
  Carnegie Mellon University, IBM</div>
<br>
\end{rawhtml}
}


\PageHead{Downloading \Bonmin}
\begin{PageSummary}
\PageSection{Obtaining \Bonmin}{sec:obtain}
\PageSection{Obtaining required third party code}{sec:obtain_3rd}
\end{PageSummary}

\begin{quickref}
\quickcitation{Bonmin Wiki Pages}{\linkCoin Bonmin}
\quickcitation{subversion web page}{http://subversion.tigris.org/}
\quickcitation{Using subversion on windows}{http://www.coin-or.org/faqs.html\#q4}
\quickcitation{Linear Algebra PACKage}{http://www.netlib.org/lapack/}
\quickcitation{Basic Linear Algebra Subroutines}{http://www.netlib.org/blas/}
\quickcitation{Harwell Subroutine Library}{http://www.cse.clrc.ac.uk/nag/hsl/contents.shtml}
\quickcitation{Ampl Solver Library}{http://www.ampl.com}
\end{quickref}

\PageTitle{Obtaining \Bonmin}{sec:obtain}



The \Bonmin\ package consists of the source code for the \Bonmin\
project but also source code from other \COINOR\ projects:
\begin{itemize}
\item \BuildTools
\item \Cbc
\item \Cgl
\item \Clp
\item \CoinUtils
\item \Ipopt
\item \Osi
\end{itemize}

When downloading the \Bonmin\ package you will download the source code for all these and
libraries of problems to test the codes.\\


You can obtain the \Bonmin\ package by using
\link{http://subversion.tigris.org/}{subversion}.

In Unix\footnote{UNIX is a registered trademark of The Open
Group.}-like environments, to download the code in a sub-directory, say {\tt coin-Bonmin} issue the following
command:
\break

\begin{colorverb}
 \noindent {\tt  svn co
https://projects.coin-or.org/svn/Bonmin/trunk~coin-Bonmin }
\end{colorverb}

This copies all the necessary COIN-OR files to compile \Bonmin\ to
{\tt coin-Bonmin}. To download \Bonmin\ using svn on Windows,
follow the instructions provided at
\link{http://www.coin-or.org/faqs.html\#q4}{COIN-OR}.

\subsectionH{Obtaining required third party code}{sec:obtain_3rd}
\Bonmin\ needs a few external packages which are not included in the \Bonmin\ package:
\begin{itemize}
\item Lapack (Linear Algebra PACKage)
\item Blas (Basic Linear Algebra Subroutines)
\item the sparse linear solver MA27 from the Harwell Subroutine Library and optionally (but strongly recommended) MC19 to enable automatic scaling in \Ipopt.
\item optionally ASL (the Ampl Solver Library), to be able to use \Bonmin\ from Ampl.
\end{itemize}

Since these third-party software modules are released under licenses
that are incompatible with the CPL, they cannot be included for
distribution with \Bonmin\ from COIN-OR, but you will find scripts
to help you download them in the subdirectory {\tt ThirdParty} of
the \Bonmin\ distribution\footnote{In most Linux distribution and
CYGWIN, Lapack and Blas are available as prebuilt binary packages in
the distribution (and are probably already installed on your
machine).}. For details on how to obtain these package, refer to the
instructions in
\link{http://www.coin-or.org/Ipopt/documentation/node13.html}{Section 2.2} of the Ipopt manual.\\


\PageHead{Installation}
\StartPageSummary
\PageSection{Installing \Bonmin}{sec:install}
\PageSection{Algorithms}{Algos}
\PageSection{Building the documentation}{sec:ref_man}
\PageSection{Running the test programs}{sec:test}
\EndPageSummary
\NavigationPanel

%\section{Layout of the \Bonmin\ package directory}

%\section{Layout of the \Bonmin\ project directory}
%The \Bonmin\ directory contains several sub-folders and files. We give here a brief overview of their content:
%\begin{itemize}
%\item {\tt Apps/} contains the sources to build the executable {\tt bonmin},
%\item {\tt BCP/} (undocumented) an experimental MINLP solver for parallel environments using
%  \href{http://www.coin-or.org/documentation.html\#BCP}{BCP}\footnote{http://www.coin-or.org/documentation.html\#BCP},
%\item {\BonminBB} contains the source of the library to interface {\tt IpoptInterface}, {\tt Oa} and {\tt Cbc} and
%to carry out a branch-and-bound algorithm,
%\item {\tt CbcCopy/} contains a copy of the most recent version of
%\href{http://www.coin-or.org/Cbc}{\tt Cbc} that has been tested and fully works with \Bonmin,
%\item {\tt Doc/} contains this documentation and the reference manual
%(to generate the reference manual see Section \ref{sec:ref_man}),
%\item {\tt IpoptInterface/} contains the source for an
%\href{http://www.coin-or.org/projects.html\#OSI}{\tt OsiInterface}\footnote{http://www.coin-or.org/projects.html\#OSI}
%interface to {\tt Ipopt} ({\tt OsiInterface}'s are mainly for LP solvers, with some
%rudimentary support for MILP solvers; so it is not truly an {\tt OsiInterface},
%but it implements sufficient
%functions for carrying out branch-and-bound),
%\item {\tt OaInterface} contains the necessary elements for interfacing outer-approximation inside {\tt Cbc} using an
%{\tt IpoptInterface},
%\item {\tt Test} contains some test problems (see Section \ref{sec:test}),
%\item {\tt Makefile} is the main Makefile,
%\item {\tt Makefile.conf} contains the information for configuring the build process of \Bonmin,
%\item {\tt Makefile.lib} contains general building rules.
%\end{itemize}

\PageTitle{Installing \Bonmin}{sec:install}
The build process for \Bonmin\ should be fairly automatic as it uses
\link{http://sources.redhat.com/autobook/autobook/}{GNU autotools}.
  It has been successfully compiled and run on the following platforms:
\begin{itemize}
\item Linux using g++ version 3.4 and 4.0
\item Windows using version Cygwin 1.5.18
\item Mac OS X using gcc 3.4 and 4.0
\end{itemize}

For Cygwin and OS X some specific setup has to be done prior to instalation. These step are described on the wiki pages of {\tt Bonmin}  \footlink{https://projects.coin-or.org/Bonmin/wiki/CygwinInstall}{CygwinInstall} and \footlink{https://projects.coin-or.org/Bonmin/wiki/OsxInstall}{OsxInstall}.


\Bonmin\ is compiled and installed using the commands:
\begin{verbatim}

./configure -C
make
make install

\end{verbatim}

This installs the executable {\tt bonmin} in {\tt coin-Bonmin/bin}. In what follows, we assume
that you have put the executable {\tt bonmin} on your path.

The {\tt configure} script attempts to find all of the machine specific settings (compiler, libraries,...)
necessary to compile and run the code. Although {\tt configure} should find most of the standard
ones, you may have to manually specify a few of the settings.
The options for the configure script can be found by issuing the command
\begin{verbatim}

./configure --help

\end{verbatim}
For a more in depth description of these options,
the reader is invited to refer to the COIN-OR {\tt BuildTools} \footlink{\linkCoin BuildTools}{trac page}.

\subsectionH{Specifying the location of {\tt Cplex} libraries}{}
If you have {\tt Cplex} installed on your machine, you may want to use it
as the Mixed Integer Linear Programming subsolver in {\tt B-OA} and {\tt B-Hyb}.
To do so you have to specify the location of the header files and libraries.
You can either specify the location of the header files directory by passing it as an
argument to the configure script or by writing it into a {\tt config.site} file.\\

In the former case, specify the location of the {\tt Cplex} header files by using the
argument {\tt --with-cplexincdir} and the location of the
{\tt Cplex } library with {\tt --with-cplexlib} (note that on the Linux platform you will also
need to add {\tt -lpthread} as an argument to {\tt --with-cplexlib}).\\

For example, on a Linux machine if {\tt Cplex} is installed in {\tt /usr/ilog}~, you would
invoke configure with the arguments as follows:
\begin{verbatim}

./configure --with-cplex-incdir=/usr/ilog/cplex/include/ilcplex \
  --with-cplex-lib="/usr/ilog/cplex/lib/libcplex.a -lpthread"
 \end{verbatim}
In the latter case, put a file called {\tt config.site} in a subdirectory named
{\tt share} of the installation directory (if you do not specify an alternate
installation directory to the {\tt configure} script with the {\tt --prefix}
argument, the installation directory is the directory where you execute the
{\tt configure} script). To specify the location of {\tt Cplex}~, insert the
following lines in the {\tt config.site} file:
 \begin{verbatim}

 with_cplex_lib="/usr/ilog/cplex/lib/libcplex.a -lpthread"
 with_cplex_incdir="/usr/ilog/cplex/include/ilcplex"
 \end{verbatim}
 (You will find a {\tt config.site} example in the subdirectory {\tt BuildTools} of {\tt coin-Bonmin}.)

\subsectionH{Compiling \Bonmin\ in a external directory}{sec:vpath}
It is possible to compile \Bonmin\ in a directory different from {\tt coin-Bonmin}.
This is convenient if you want to have several executables compiled for different a
rchitectures or have several executables compiled with different options
(debugging and production, shared and static libraries).\\

To do this just create a new directory, for example {\tt Bonmin-build} in the parent directory of
{\tt coin-Bonmin} and run the configure command from {\tt Bonmin-build}:
\begin{verbatim}

../Bonmin-0.1/configure -C

\end{verbatim}
This will create the makefiles in {\tt coin-Bonmin}, and
you can then compile with the usual {\tt make} and {\tt make install}
(in {\tt Bonmin-build}).

\subsectionH{Building the documentation}{sec:ref_man}
The documentation for \Bonmin\ consists of a users' manual (this document) and a reference manual.
You can build a local copy of the reference manual provided that you have Latex
and Doxygen installed on your machine. Issue the command {\tt make
doxydoc} in {\tt coin-Bonmin}. It calls Doxygen to build a copy of the
reference manual. An html version of the reference manual can then
be accessed in {\tt doc/html/index.html}.

%You can also build a pdf
%version of the reference manual by issuing the command {\tt make
%refman.pdf} ({\tt refman.pdf} is placed in the {\tt doc} subdirectory).

\subsectionH{Running the test programs}{sec:test}
By issuing the command {\tt make test}~, you build and run the automatic test program for \Bonmin.


\PageHead{Running \Bonmin}
\StartPageSummary
\PageSection{On an .nl file}{sec:run_nl}
\PageSection{From Ampl}{sec:run_ampl}
\PageSection{From a C++ Programm}{sec:run_cpp}
\EndPageSummary
\NavigationPanel

\begin{quickref}
\quickcitation{Writing \texttt{\bf .nl} files. D.M.~Gay.}{\BibPage #Gay}
\quickcitation{AMPL: A Modeling Language for Mathematical
Programming, Second Edition, Duxbury Press Brooks Cole Publishing Co., 2003. R.~Fourer and D.M.~Gay and B.W.~Kernighan.}{\BibPage #AMPL}
\end{quickref}


\PageTitle{Running \Bonmin}{sec:run}
\Bonmin\ can be run
\begin{itemize}
\item [(i)] from a command line on a {\tt .nl} file
(see \mycite{Gay}{Gay2005}),
\item [(ii)] from the modeling language \footlink{http://www.ampl.com}{\tt Ampl} (see
\mycite{AMPL}{Fourrer2003}) or from \footlink{http://www.gams.com/}{Gams} provided
that you have a valid {\tt Ampl} license, and
\item [(iii)] by invoking it from a C/C++ program.
\item[(iv)] remotely through the \footlink{http://neos.mcs.anl.gov/neos}{NEOS} web interface.
\end{itemize}

In \latexhtml{the subsections  that follow}{this page}, we give some details about the
various ways to run \Bonmin.

\subsectionH{On a {\tt .nl} file}{sec:run_nl}
\Bonmin\ can read a {\tt .nl} file which could be generated by {\tt
Ampl} (for example {\tt mytoy.nl} in the {\tt
Bonmin-dist/Bonmin/test} subdirectory). The command line takes just
one argument which is the name of the {\tt .nl} file to be
processed.

For example, if you want to solve {\tt mytoy.nl}, from the {\tt
Bonmin-dist} directory, issue the command:

\begin{colorverb}
\begin{verbatim}

bonmin test/mytoy.nl

\end{verbatim}
\end{colorverb}

\subsectionH{From {\tt Ampl}}{sec:run_ampl}
To use \Bonmin\ from {\tt Ampl} you just need to have the directory where the
{\tt bonmin} executable is in your {\tt \$PATH} and to issue the
command

\begin{colorverb}
\begin{verbatim}

option solver bonmin;

\end{verbatim}
\end{colorverb}

in the {\tt Ampl} environment. Then the next {\tt solve} will
use \Bonmin\ to solve the model loaded in {\tt Ampl}.
After the optimization is finished, the values of the variables in the best-known
or optimal solution can be accessed in {\tt Ampl}. If the optimization is interrupted
with {\tt <CTRL-C>} the best known solution is accessible (this feature is not available in Cygwin).\\

A simple {\tt Ampl} example model follows:

\begin{colorverb}
\begin{verbatim}

   # An Ampl version of toy

   reset;

   var x binary;
   var z integer >= 0 <= 5;
   var y{1..2} >=0;
   minimize cost:
       - x - y[1] - y[2] ;

   subject to
       c1: ( y[1] - 1/2 )^2 + (y[2] - 1/2)^2 <= 1/4 ;
       c2: x - y[1] <= 0 ;
       c3: x + y[2] + z <= 2;

   option solver bonmin; # Choose BONMIN as the solver (assuming that
                         # bonmin is in your PATH

   solve;                # Solve the model
   display x;
   display y;

\end{verbatim}
\end{colorverb}

(This example can be found in the subdirectory {\tt Bonmin/examples/amplExamples/} of
the \Bonmin\ package.)

Branching priorities, branching directions and pseudo-costs can be passed using {\tt Ampl} suffixes.
The suffix for branching priorities is {\tt "priority"} (variables with a higher priority
will be chosen first for branching),
for branching direction is {\tt "direction"} (if direction is $1$ the $\geq$ branch
is explored first, if direction is $-1$ the $\leq$ branch is explored first), for up
and down pseudo costs {\tt "upPseudoCost"} and {\tt "downPseudoCost"} respectively
(note that if only one of the up and down pseudo-costs is set in the {\tt Ampl} model it will
be used for both up and down).\\

For example, to give branching priorities of $10$ to variables {\tt y} and 1 to variable {\tt x}
and to set the branching directions to explore the upper branch first for all variables
in the simple example given, we add before the call to solve:
\begin{colorverb}
\begin{verbatim}

suffix priority IN, integer, >=0, <= 9999;
y[1].priority := 10;
y[2].priority := 10;
x.priority := 1;

suffix direction IN, integer, >=-1, <=1;
y[1].direction := 1;
y[2].direction := 1;
x.direction := 1;

\end{verbatim}
\end{colorverb}

SOS Type-1 branching is also available in \Bonmin\ from {\tt Ampl}. We
follow the conventional way of doing this with suffixes.
Two type of suffixes should be declared:

\begin{colorverb}
\begin{verbatim}
suffix sosno IN, integer, >=1;  # Note that the solver assumes that these
                                #   values are positive for SOS Type 1
suffix ref IN;
\end{verbatim}
\end{colorverb}

Next, suppose that we wish to have variables

\begin{colorverb}
\begin{verbatim}
var X {i in 1..M, j in 1..N} binary;
\end{verbatim}
\end{colorverb}
and the ``convexity'' constraints:

\begin{colorverb}
\begin{verbatim}
subject to Convexity {i in 1..M}:
   sum {j in 1..N} X[i,j] = 1;
\end{verbatim}
\end{colorverb}

(note that we must explicitly include the convexity constraints in the {\tt Ampl} model).

Then after reading in the data, we set the suffix values:
\begin{colorverb}
\begin{verbatim}

# The numbers `val[i,j]' are chosen typically as
#     the values `represented' by the discrete choices.
let {i in 1..M, j in 1..N} X[i,j].ref := val[i,j];

# These identify which SOS constraint each variable belongs to.
let {i in 1..M, j in 1..N} X[i,j].sosno := i;
\end{verbatim}
\end{colorverb}

\subsubsection{From {\tt Gams} using {\tt Ampl}}
The modeling system {\tt Gams} offers the possibility to solve {\tt Gams} models using any
{\tt Ampl} capable solver. The {\tt Gams}/{\tt Ampl} link comes free with the {\tt Gams}
system, but users must have a licensed {\tt Ampl} on their machine.

To be able to use \Bonmin\ with {\tt Gams}, you must compile the COIN-OR libraries as static.
This can be done by passing the option {\tt --enable-static} to the configure file.

Detailed instruction for using {\tt Gams}/{\tt Ampl} can be found
on \footlink{http://www.gams.com/solvers/gamsampl.pdf}{{\tt Gams} website}.\\

Note that user set branching priorities and SOS Type-1 branching is not
available from this interface.

\subsectionH{From a C/C++ program}{sec:run_cpp}
\Bonmin\ can also be run from within a C/C++ program if the user codes
the functions to compute first- and second-order derivatives.
An example of such a program is available in the subdirectory {\tt CppExample} of
the {\tt examples} directory. For further explanations, please refer to the reference manual.


\input{BOUM_options_set}
%\section{{\Bonmin} output}
%\Bonmin uses a number of optimization tools ({\
%tt Ipopt}, {\tt Cbc}, {\tt Clp}, \ldots) which all have there own output.
%As a consequence the output is very diverse and may be a little difficult to read.\\
%
%The output level for each of the building blocks of \Bonmin can be set through options (see Appendix \ref{app:opt_loglevel})
%Here we briefly describes the output given at the different log level for each of those building blocks.
%\subsection{BB\_log_level}
%This corresponds to the output from the main branch-and-bound process in B-BB, B-QG, B-Hyb
%(this is not available in the B-OA).
%The output here comes from {\tt Cbc}, and each line of output has the prefix {\tt Cbc} and is followed by a number indicating the message type.
%
%\subsection{
\input{BOUM_bib}
\appendix
\latexhtml{\section{List of \Bonmin\ options}
\label{sec:optList}
\input{BOUM_options_list}}


\end{document}
